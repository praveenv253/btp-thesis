\chapter{INTRODUCTION}
\label{chap:intro}

\section{Orthogonal Frequency Division \\ Multiplexing}

\subsection{Digital communication}
\label{subsec:digi-comm}

When transmitting digital data over a real (analog) channel, we typically first
convert the bit stream into a stream of complex symbols, using some kind of
modulation scheme. Following this, the symbols are pulse-shaped and upconverted
to the carrier frequency, real and imaginary parts of the symbols being encoded
in the in-phase and quadrature components of the carrier.

We can retrieve the complex symbols on the receiver side by performing matched
filtering followed by suitably sampling the analog data. In such a scenario,
the effects of the channel can also be viewed as purely digital operations on
the complex data stream. We can, therefore, work at the level of digital
abstraction, where we only deal with a discrete sequence of complex symbols.

\subsection{Channel effects}

The modulation scheme used will decide how robust the message will be against
channel effects. A simple but effective digital channel model has an impulse
response to model inter-symbol interference (caused due to multi-path effects
in wireless transmission, for example) and an addition of white gaussian noise.
Simple modulation schemes such as QAM do not fare well in such a channel,
however, due to ISI.

In order combat the effect of ISI in such cases, we would need to employ
computationally intensive decoding techniques such as the Viterbi algorithm.
Alternatives are to use sub-optimal equalization filters such as the
zero-forcing equalizer, but depending upon how many taps the channel has, these
filters could turn out to be extremely long, which once again, makes them
computationally intensive.

\subsection{OFDM solves the ISI problem}

OFDM is a modulation scheme that effectively solves the problem of ISI without
introducing heavy computational complexity. It does this by placing a block of
complex data symbols on adjacent narrow-band `subcarriers' in the frequency
domain. The transmitted data block is the inverse-FFT of the frequency domain
block. On the receiver side, each data block is converted back into frequency
domain by taking the FFT of the block.

Each block of data, after the inverse-FFT is performed, is prepended with a
cyclic prefix, which is a copy of the last few symbols. The length of this
cyclic prefix is equal to the number of taps in the channel's impulse response.
The cyclic prefix ensures that the effects of ISI remain limited to the same
data block and guards against pollution of symbols from adjacent data blocks.

With this framework, the ISI of the channel gets expressed in frequency domain
as frequency-selective fading, with each subcarrier seeing a different gain.
If we have a model of the channel in frequency domain, (by taking the FFT of
the channel taps), then we can reverse the effects of ISI completely by simply
dividing each received subcarrier symbol by the channel response at that
subcarrier.

%%%%%%%%%%%%%%%%%%%%%%%%%%%%%%%%%%%%%%%%%%%%%%%%%%%%%%%%%%%%%%%%%%%%%%%%%%%%%%%

\section{The USRP and the UHD API}

The USRP platform is a computer-hosted software-defined radio. In one sentence,
the USRP allows you to work at the level of digital abstraction alluded to in
subsection~\ref{subsec:digi-comm}. The device is connected to the computer
using either an ethernet cable or a USB 3.0 cable (depending upon the USRP
model used), and one can interface it to one's program by using the open source
UHD API.

Different USRP boards can work at different frequencies and have different
rates. The devices have the capability to act as a full transceiver. But at
least two distinct devices are required in order to perform experiments where
data is to be transmitted and received.

The UHD API can be used to interface with the USRP from a C++ program. The API
provides classes and methods to set up the USRP device with the desired
frequency, rate and gain settings. One can instantiate a transmit- or receive-
streamer to send or receive data in the form of a complex data stream.

In this project, the UHD API has been made use of extensively. The use of the
API has been restricted to two modules, namely the USRP Transmitter module and
the USRP Receiver module. These are described in
subsections~\ref{subsec:usrp-transmitter} and~\ref{subsec:usrp-receiver}.
