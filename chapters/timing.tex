\chapter{TIMING ANALYSIS}
\label{chap:timing}

\section{Packet detection}

Timing analysis is used to determine the point in time where the packet starts.
This is also a method to recognize \emph{whether} a packet has arrived or not.
In the current scheme, the packet detection module uses the Schmidl and Cox %(TODO: Put citation here)
method for timing analysis.

\subsection{The Schmidl and Cox algorithm}

For the receiver to be able to pick out a packet from ambient noise and
interference, the transmitted packet must be designed for detection. All
transmitted packets have a \emph{preamble}, which consists of two identical
halves. Each half is a pseudo-random-number sequence. The correlation of each
half with with other (independent and hence uncorrelated) signals is expected
to be low. However, its correlation with the other half will be high. Moreover,
this property is well-maintained even when the preamble is passed through an
ISI channel with additive white gaussian noise.

In other words, we can detect the start of a packet by correlating two adjacent
windows, each having half the size of the preamble, with each other. The point
where this correlation value becomes high can be taken to be the start of the
packet.

% TODO: Insert plot of preamble (real and imaginary) here
% TODO: Insert a picture of the two windows correlating received data
% TODO: Insert a picture of the metric over the data

\subsection{Running correlation on the receive buffer}

In order to efficiently perform a running correlation of adjacent windows over
an entire receive buffer, we minimize the number of computations performed. On
moving one step, we add the latest correlation point and subtract the oldest
correlation point.

Let $\ue{x}$ and $\ue{y}$ be complex vectors corresponding to the symbols in
the left half-window and the right-half-window respectively. Let $n$ be the
half-window size. After moving one time step, let these windows be denoted by
the complex vectors $u$ and $v$ respectively.  We thus have $u_i = x_{i+1}$ and
$v_i = y_{i+1}$ for $i = {1, 2, \ldots n-1}$.

Let $c_{old}$ be the correlation of $x$ with $y$, and $c_{new}$ be the
correlation of $u$ with $v$. That is,
$$ c_{old} = \corr{\ue{x}}{\ue{y}} = \sum_1^n{\corr{x_i}{y_i}} $$
$$ c_{new} = \corr{\ue{u}}{\ue{v}} = \sum_1^n{\corr{u_i}{v_i}} $$
where $^*$ denotes complex conjugation. We can then write $c_{new}$ in terms of
$c_{old}$ as follows:
\begin{align}
	c_{new} &= \sum_1^{n-1}{\corr{u_i}{v_i}} + \corr{u_n}{v_n} \\
	        &= \sum_2^n{\corr{x_i}{y_i}} + \corr{u_n}{v_n} \\
	        &= \sum_1^n{\corr{x_i}{y_i}} - \corr{x_1}{y_1} + \corr{u_n}{v_n} \\
	        &= c_{old} - \corr{x_1}{y_1} + \corr{u_n}{v_n}
\end{align}

This way, once we have acquired the correlation of the first $n$ symbols, the
correlation of subsequent windows is an $\mathcal{O}(1)$ process.

